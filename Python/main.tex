\documentclass{cheatsheet}
\usepackage{blindtext}


\usepackage{listings}
\usepackage{bm}

\doctitle{Python}
\author{Leonardo Saurwein - lsaurwein@student.ethz.ch}
\setColor{darkgray}


\begin{document}
\section{General}
    \subsection{Operators}
    \textbf{x+y} b: sum of x and y \hspace{1cm} a \textbf{-} b: minus \hspace{1cm} a \textbf{*} b: times \\
    a \textbf{**} n: a to the power of 7 \\
    a \textbf{/} b: division, return a float \\
    a \textbf{//} b: floor division discards the fractional part \\
    a \textbf{\%} b: the operator \% returns the remainder of the division \\
    \textbf{\_} :console, the last printed expression is assigned to the variable \_ \\
    \textbf{abs(x)}: absolute value or magnitude of x \\
    \textbf{int(x)}: x converted to integer \\
    \textbf{float(x)}: x converted to floating point \\
    \textbf{complex(re, im)}: complex number, re: real, im: imaginary \\
    \textbf{c.conjugate()}: conjugate of the complex number c
    bm
    \subsection{Bitwise Operations}
    TODO

    \subsection{Boolean operators}
    \textbf{x or y}: if x is false, then y, else x \\
    \textbf{x and y}: if x is false, then x, else y \\
    \textbf{not x}: if x is false, then True, else False \\
    \subsection{Comparisons}
    \boldsymbol{$<$}: strictly less than \\
    \textbf{$<=$}: less than or equal \\
    \textbf{$>$}: strictly greater than \\
    \textbf{$>=$}: greater than or equal \\
    \textbf{$==$}: equal \\
    \textbf{$!=$}: not equal \\
    \textbf{is}: object identity \\
    \textbf{is not}: negated object identity

    \subsection{Strings}
    \textbf{s = 'a' or "a"}: single quotes strings \\
    \textbf{'doesn$\backslash$'t'}: use $\backslash$' to escape the single quote \\
    \textbf{"doesn't"}: or use double quotes instead \\
    \textbf{$\backslash$n}: go to the next line \hspace{0.3cm} \textbf{$\backslash$t}: tab space \hspace{0.3cm} \textbf{$\backslash$r}: carriage return \\
    \textbf{r"..."}: raw strings \\
    \textbf{"""..."""}: string literals on multiple lines, use $\backslash$ to prevent $\backslash$n \\
    \textbf{+} and \textbf{*}: sum or multiply strings \\
    \textbf{a[n]}: to access n index of a (start from 0) \\
    \textbf{a[-k]}: to access n-k index of a (start from n+1) \\
    \textbf{a[i:j]}: range from i to j, leave black to get first or last \\
    \textbf{strings are immutable (don't support index assignment)} \\
    Methods (all return a copy): \\
    \textbf{len(a)}: return the length of the string \\
    \textbf{s.capitalize()}: first character capitalized and the rest lowercased \\
    \textbf{s.casefold()}: casefolded copy of the string.
    \textbf{s.center(w, c)}: centered in a string of length w, fill with c \\
    \textbf{s.count(i)}: count ammount of substring i in str \\
    \textbf{s.encode()}: encode the string, default: utf-8 \\
    \textbf{s.endswith(s)}: true if s end with s, false otherwise \\
    \textbf{s.startswith(s)}: true if s start with s, false otherwis \\
    \textbf{s.expandtabs(n)}: expand the tab ($\backslash$t) \\
    \textbf{s.find(i)}: return smallest index of i in s, -1 if not found \\
    \textbf{s.index(i)}: like find but raise an error \\
    \textbf{s.isalnum()}: true if alphanumeric \\
    \textbf{s.isalpha()}: true if all alphabetic \\
    \textbf{s.isascii()}: true if ascii \\
    \textbf{s.isdecimal()}: true if all decimal \\
    \textbf{s.isdigit()}: true if all digit \\
    \textbf{s.islower()}: true if all lowercase \\
    \textbf{s.lower()}: all cased characters converted to lowercase \\
    \textbf{s.isupper()}: true if all uppercase \\
    \textbf{s.upper()}: all cased characters converted to uppercase \\
    \textbf{s.isnumeric()}: true if all numeric character \\
    \textbf{s.isspace()}: true if only whitespace characters \\
    \textbf{s.istitle()}: true if title first upper rest lower \\
    \textbf{s.title()}: convert s to a title \\
    \textbf{s.strip([c])}: strip c from left and right, black by deafult \\
    \textbf{s.lstrip([c])}: --> remove c from left, blank by default \\
    \textbf{s.rstrip([c])}: <-- remove c from right, blank by deafult \\
    \textbf{s.removeprefix(i)}: remove prefix i\\
    \textbf{s.removesuffix(i)}: remove sufflix i \\
    \textbf{s.replace(old, new, count)}: replace all old with new \\
    \textbf{s.split(sep)}: split on sep return a list \\
    \textbf{s.splitlines()}: split on \textbf{$\backslash$n} return a list \\
    \textbf{s.swapcase()}: convert upper to lower and viceversa

    \subsection{Tuples and Sets}
    TODO

    \subsection{Lists}
    \textbf{l = [a,b,c]}: list, support index like strings \\
    \textbf{l[:]}: return a copy of l \\
    \textbf{l[n] = k}: set n index to k, work also with a slice (l[i:j]) \\
    \textbf{[i for i in a]}: concise way to create lists \\
    \textbf{del l[n]}: delete n index, work also with interval \\
    Data structure: \\
    \textbf{len(l)}: return the length of the list \\
    \textbf{min(l)}: return smallest item of s \\
    \textbf{max(l)}: return largest item of s \\
    \textbf{l.append(i)}: add i to the end of the list (a[len(a):] = [x]) \\
    \textbf{l.extend(it)}: appending all the items (a[len(a):] = it) \\
    \textbf{l.insert(i, x)}: insert at i, x all the value shift \\
    \textbf{l.remove(x)}: remove first item that equal x. error if don't exist\\
    \textbf{l.pop([i])}: remove index i, if blank the last. return the value \\
    \textbf{l.clear()}: remove all the elements (del a[:])\\
    \textbf{l.index(x)}: return index of x. error if don't exist \\
    \textbf{l.count(x)}: return the ammount of x \\
    \textbf{l.sort()}: sort the list \\
    \textbf{l.reverse()}: flip the list ([::-1]) \\
    \textbf{l.copy()}: return a copy of the list ([:])
    \subsection{Dictionaries}
    
\section{Flow Tools}


    \subsection{if Statements}
    Use: \\
    if (condition): \\
    elif (condition): \\
    else: 

    \subsection{Classes Special Attributes}
    \textbf{\_\_dict\_\_}: A dictionary or other mapping \\
    \textbf{\_\_class\_\_}: The class to which a class instance belongs \\
    \textbf{\_\_bases\_\_}: The tuple of base classes of a class object.\\
    \textbf{\_\_name\_\_}: The name of the class, function, method, descriptor, or generator instance.\\
    \textbf{\_\_\_\_}: \\
    \textbf{\_\_\_\_}: \\
    \textbf{\_\_\_\_}: \\
    \textbf{\_\_\_\_}: \\
    \textbf{\_\_\_\_}: \\
    \textbf{\_\_\_\_}: \\
    \textbf{\_\_\_\_}: \\
    \textbf{\_\_\_\_}: \\
    \textbf{\_\_\_\_}: \\

\section{Libraries}
    \subsection{Basics}
    \subsection{Os}
    \subsection{Math}
    \subsection{Random}
    \subsection{Statistics}
    \subsection{Matplotlib}
    \subsection{Numpy}
    \subsection{Pandas}
    \subsection{Datetime}
    \subsection{Timeit}
    \subsection{Pygame}
    \subsection{Threading}
    \subsection{Requests}
    \subsection{Flask}
    

\end{document}