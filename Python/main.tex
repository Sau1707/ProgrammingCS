\documentclass{cheatsheet}
\usepackage{blindtext}
\usepackage{listings}
\usepackage{bm}
\usepackage[T1]{fontenc}

\doctitle{Python}
\author{Leonardo Saurwein - lsaurwein@student.ethz.ch}
\setColor{darkgray}


\begin{document}
\section{General}
    \subsection{Operators}
    Define a number variable:
    \pyBox{
        i = 0 # Create an integer^^J
        f = 0.00 # Create a float
    }
    \makeDef{x + y}{sum of x and y}
    \makeDef{x - y}{subtraction of x and y}
    \makeDef{x * y}{x times y}
    \makeDef{a ** n}{a to the power of n}
    \makeDef{a / b}{division, return a float}
    \makeDef{a // b}{floor division, discard the fractional part}
    \makeDef{a \% b}{the operator \% returns the remainder of the division}
    \makeDef{\_}{console, last printed expression is assigned to the variable \_}
    \makeDef{abs(x)}{absolute value or magnitude of x}
    \makeDef{int(x)}{x converted to integer}
    \makeDef{float(x)}{x converted to floating point}
    \makeDef{complex(re, im)}{complex number, re: real, im: imaginary}
    \makeDef{c.conjugate()}{conjugate of the complex number c}

    %% Todo 
    %\subsection{Bitwise Operations}

    \subsection{Operators}
    \textbf{Comparison:} \\
    \makeDef{<}{strictly less than}
    \makeDef{<=}{less than or equal}
    \makeDef{>}{strictly greater than}
    \makeDef{>=}{greater than or equal}
    \makeDef{==}{equal}
    \makeDef{!=}{not equal}
    \makeDef{is}{object identity}
    \makeDef{is not}{negated object identity}
    \vspace{0.1cm}
    \textbf{Logical:} \\
    \makeDef{and}{true if both true}
    \makeDef{or}{true if only one true}
    \makeDef{not}{reverse the result}
    \vspace{0.1cm}
    \textbf{Identity:} \\
    \makeDef{is}{true if both variable are the same object}
    \makeDef{is not}{true if both variable are not the same object}
    \vspace{0.1cm}
    \textbf{Membership:} \\
    \makeDef{in}{true if the value is in the object}
    \makeDef{not in}{true if value not in object}


    \subsection{Strings}
    Define a string variable:
    \pyBox{
        s = "" or '' # Create an empty string
    }
    \makeDef{"doesn't"}{mix to use single quote}
    \textbf{$\backslash$n}: go to the next line \hspace{0.3cm} \textbf{$\backslash$t}: tab space \hspace{0.3cm} \textbf{$\backslash$r}: carriage return \\
    \makeDef{r"..."}{raw strings}
    \makeDef{"""..."""}{string literals on multiple lines, use $\backslash$ to prevent $\backslash$n}
    \makeDef{+, *}{sum or multiply strings}
    \vspace{0.2cm}
    \textbf{Strings are immutable (don't support index assignment)} \\
    \makeDef{a[n]}{to access n index of a (start from 0)}
    \makeDef{a[-k]}{to access n-k index of a (start from n+1)}
    \makeDef{a[i:j]}{range from i to j, leave black to get first or last}
    \vspace{0.2cm}
    \textbf{Methods (all return a copy of the string):}\\
    \makeDef{len(a)}{return the length of the string}
    \makeDef{s.capitalize()}{first character capitalized and the rest lowercased}
    \makeDef{s.casefold()}{casefolded copy of the string.}
    \makeDef{s.center(w, c)}{centered in a string of length w, fill with c}
    \makeDef{s.count(i)}{count ammount of substring i in str}
    \makeDef{s.encode()}{encode the string, default: utf-8}
    \makeDef{s.endswith(s)}{true if s end with s, false otherwise}
    \makeDef{s.startswith(s)}{true if s start with s, false otherwis}
    \makeDef{s.expandtabs(n)}{expand the tab ($\backslash$t)}
    \makeDef{s.find(i)}{return smallest index of i in s, -1 if not found}
    \makeDef{s.index(i)}{like find but raise an error}
    \makeDef{s.isalnum()}{true if alphanumeric}
    \makeDef{s.isalpha()}{true if all alphabetic}
    \makeDef{s.isascii()}{true if ascii}
    \makeDef{s.isdecimal()}{true if all decimal}
    \makeDef{s.isdigit()}{true if all digit}
    \textbf{s.islower()}: true if all lowercase \\
    \textbf{s.lower()}: all cased characters converted to lowercase \\
    \textbf{s.isupper()}: true if all uppercase \\
    \textbf{s.upper()}: all cased characters converted to uppercase \\
    \textbf{s.isnumeric()}: true if all numeric character \\
    \textbf{s.isspace()}: true if only whitespace characters \\
    \textbf{s.istitle()}: true if title first upper rest lower \\
    \textbf{s.title()}: convert s to a title \\
    \textbf{s.strip([c])}: strip c from left and right, black by deafult \\
    \textbf{s.lstrip([c])}: --> remove c from left, blank by default \\
    \textbf{s.rstrip([c])}: <-- remove c from right, blank by deafult \\
    \textbf{s.removeprefix(i)}: remove prefix i\\
    \textbf{s.removesuffix(i)}: remove sufflix i \\
    \textbf{s.replace(old, new, count)}: replace all old with new \\
    \textbf{s.split(sep)}: split on sep return a list \\
    \textbf{s.splitlines()}: split on \textbf{$\backslash$n} return a list \\
    \textbf{s.swapcase()}: convert upper to lower and viceversa

    \subsection{Tuples and Sets}
    Define a set and tuple
    \pyBox{
        t = () # Create an empty tuple^^J
        s = set() # Create an empty set
    }

    \subsection{Lists}
    Define a list
    \pyBox{
        l = [] # Create an empty list
    }
    \textbf{Lists are mutable (support index assignment)} \\
    \makeDef{l[:]}{return a copy of l}
    \makeDef{l[n] = k}{set n index to k, work also with a slice (l[i:j])}
    \makeDef{[i for i in a]}{concise way to create lists}
    \makeDef{del l[n]}{delete n index, work also with interval}
    \vspace{0.2cm} 
    \textbf{Data structure:} \\
    \textbf{len(l)}: return the length of the list \\
    \textbf{min(l)}: return smallest item of s \\
    \textbf{max(l)}: return largest item of s \\
    \textbf{l.append(i)}: add i to the end of the list (a[len(a):] = [x]) \\
    \textbf{l.extend(it)}: appending all the items (a[len(a):] = it) \\
    \textbf{l.insert(i, x)}: insert at i, x all the value shift \\
    \textbf{l.remove(x)}: remove first item that equal x. error if don't exist\\
    \textbf{l.pop([i])}: remove index i, if blank the last. return the value \\
    \textbf{l.clear()}: remove all the elements (del a[:])\\
    \textbf{l.index(x)}: return index of x. error if don't exist \\
    \textbf{l.count(x)}: return the ammount of x \\
    \textbf{l.sort()}: sort the list \\
    \textbf{l.reverse()}: flip the list ([::-1]) \\
    \textbf{l.copy()}: return a copy of the list ([:])

    \subsection{Dictionaries}
    Definition of a empty dictionary
    \pyBox{
        d = \{\}
    }
    \makeDef{list(d)}{return a list of keys}
    \makeDef{sorted(d)}{return all the keys sorted}
\section{Flow Tools}
    \subsection{if Statements}
    
    \subsection{loop}
    \textbf{Statement} \\
    \makeDef{continue}{skip the rest of the code but continue the loop}
    \makeDef{break}{stop the loop}
    \vspace{0.1cm}
    While loop:
    \pyBox{
        while (statement):^^J
        else: #when loop finish without a break
    }
    \vspace{0.1cm}
    For loop:
    \pyBox{
        for (iterator): ^^J
        else: #when loop finish without a break
    }

    \subsection{Classes Special Attributes}
    \textbf{\_\_dict\_\_}: A dictionary or other mapping \\
    \textbf{\_\_class\_\_}: The class to which a class instance belongs \\
    \textbf{\_\_bases\_\_}: The tuple of base classes of a class object.\\
    \textbf{\_\_name\_\_}: The name of the class, function, method, descriptor, or generator instance.\\
    \textbf{\_\_\_\_}: \\
    \textbf{\_\_\_\_}: \\
    \textbf{\_\_\_\_}: \\
    \textbf{\_\_\_\_}: \\
    \textbf{\_\_\_\_}: \\
    \textbf{\_\_\_\_}: \\
    \textbf{\_\_\_\_}: \\
    \textbf{\_\_\_\_}: \\
    \textbf{\_\_\_\_}: \\

\section{Libraries}
    \subsection{Json}
        Import json package:
        \pyBox{
            import json
        }
        \makeDef{loads(s)}{convert json string in dictionary}
        \makeDef{dumps(d)}{convert dict in json string}
        Use: \textbf{ident=4, sort\_keys=True} to prettify
    \subsection{random}
        Import random module:
        \pyBox{
            import random
        }
        \makeDef{seed(value)}{Inizialize the random number genetator}
        \makeDef{randrange(start, stop, step)}{return a float in the range}
        \makeDef{ranint(start, stop+1)}{return an integer in the range}
        \makeDef{choice(seq)}{return a random element of the sequence}
        \makeDef{choices(seg, weights, k)}{return k element of the sequence}
        \makeDef{shuffle(seq)}{shuffle the sequence}
        \makeDef{random()}{return a random float between 0 and 1}
        \makeDef{uniform(a, b)}{return a float in the range}
        \makeDef{triangular(a, b, center)}{return a float in range with center}
    \subsection{Os}
    \subsection{Math}
    \subsection{Random}
    \subsection{Statistics}
    \subsection{Matplotlib}
    \subsection{Numpy}
    \subsection{Pandas}
    \subsection{Datetime}
    \subsection{Timeit}
    \subsection{Pygame}
    \subsection{Threading}
    \subsection{Requests}
        Import request module:
        \pyBox{
            import requests
        }
        \makeDef{get(url, params, args)}{GET request to url}
        \makeDef{post(url, data, json, args)}{POST request to url}
        \makeDef{put(url, data, args)}{PUT request to url}
        \makeDef{delete(url, args)}{DELETE request to url}
        \makeDef{head(url, args)}{HEAD request to url}
        \makeDef{patch(url, data, args)}{PATCH request to url}
    \subsection{Flask}
    

\end{document}